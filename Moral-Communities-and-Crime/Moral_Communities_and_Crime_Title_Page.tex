% !TeX program = pdfLaTeX
\documentclass[smallextended]{svjour3}       % onecolumn (second format)
%\documentclass[twocolumn]{svjour3}          % twocolumn
%
\smartqed  % flush right qed marks, e.g. at end of proof
%
\usepackage{amsmath}
\usepackage{graphicx}
\usepackage[utf8]{inputenc}

\usepackage[hyphens]{url} % not crucial - just used below for the URL
\usepackage{hyperref}
\providecommand{\tightlist}{%
  \setlength{\itemsep}{0pt}\setlength{\parskip}{0pt}}

%
% \usepackage{mathptmx}      % use Times fonts if available on your TeX system
%
% insert here the call for the packages your document requires
%\usepackage{latexsym}
% etc.
%
% please place your own definitions here and don't use \def but
% \newcommand{}{}
%
% Insert the name of "your journal" with
% \journalname{myjournal}
%

%% load any required packages here





\begin{document}

\title{Spatial clustering and violent crime near places of worship in Recife,
Brazil. Do moral communities have a spatial dimension? \thanks{\textbf{Funding:} Edivaldo Constantino das Neves Júnior received support
from the Government of Canada, through the program Emerging Leaders in
the Americas. \textbf{Conflicts of Interest:} The authors declare that
they have no conflict of interest.} }


    \titlerunning{Moral communities and crime}

\author{  Edivaldo Constantino das Neves Júnior \and  Breno Caldas de Araujo \and  Tatiane Almeida de Menezes \and  Antonio Paez \and  }


\institute{
        Edivaldo Constantino das Neves Júnior \at
     University of Sao Paulo - Brazil \\
     \email{\href{mailto:edivaldoconstantino@gmail.com}{\nolinkurl{edivaldoconstantino@gmail.com}}}  %  \\
%             \emph{Present address:} of F. Author  %  if needed
    \and
        Breno Caldas de Araujo \at
     Federal University of Pernambuco - Brazil \\
     \email{\href{mailto:brenocaldasdearaujo@gmail.com}{\nolinkurl{brenocaldasdearaujo@gmail.com}}}  %  \\
%             \emph{Present address:} of F. Author  %  if needed
    \and
        Tatiane Almeida de Menezes \at
     Federal University of Pernambuco - Brazil \\
     \email{\href{mailto:tatianedemenezes@gmail.com}{\nolinkurl{tatianedemenezes@gmail.com}}}  %  \\
%             \emph{Present address:} of F. Author  %  if needed
    \and
        Antonio Paez \at
     McMaster University - Canada \\
     \email{\href{mailto:paezha@mcmaster.ca}{\nolinkurl{paezha@mcmaster.ca}}}  %  \\
%             \emph{Present address:} of F. Author  %  if needed
    \and
    }

\date{Received: date / Accepted: date}
% The correct dates will be entered by the editor


\maketitle

\begin{abstract}
Religious tenets of the type ``thou shalt not kill'' and their
equivalents in many world religions have functioned as \emph{de facto}
social policy for determining appropriate and acceptable behavior over
the centuries. With the advent of the scientific study of religion,
there has been a growth of interest in the role of religions to operate
as moral communities. Moral communities, a concept closely related to
informal social control, are of interest in countries and regions where
formal controls are weak and ineffective. The objective of this paper is
to present a spatial analysis of Violent and Intentional Crime in the
city of Recife in Brazil, with a focus on the possible interactions
between criminal events and places of worship. Previous research into
moral communities has advocated the need for analysis at different
scales, and this analysis contributes to the literature by using
micro-level data and appropriate spatial analytical tools for spatial
point patterns. Analysis is conducted using three different types of
places of worship (Catholic, Evangelical, and Spiritist) and three types
of business establishments as controls (ice cream shops, pharmacies, and
supermarkets). The results suggest that Catholic places of worship do
not project moral communities geographically more than, say, ice cream
shops. The intensity of criminal events in the proximity of Evangelical
places of worship, in contrast, is markedly higher at short distances
than for any of our other referential events; however, at distances
between 300 and 500 m the intensity is lower. Finally, the intensity of
crime with respect to Spiritist churches, albeit lower at short
distances, tends to be significantly higher at distances between 300 and
500 m.
\\
\keywords{
        Moral communities \and
        Crime \and
        Places of worship \and
        Point pattern analysis \and
        Intensity \and
    }


\end{abstract}


\def\spacingset#1{\renewcommand{\baselinestretch}%
{#1}\small\normalsize} \spacingset{1}




\bibliographystyle{spbasic}
\bibliography{bibliography.bib}

\end{document}
